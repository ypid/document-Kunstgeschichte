\renewcommand{\BildName}{Metaphysische Landschaft}
\section{Bildbeschreibung: \enquote{\BildName}}
Staatsgalerie Stuttgart: Raum 18

\lettrine{D}{as} Bild von Yves Tanguy (1900--1955) steht unter dem Titel \enquote{\BildName} und wurde 1935 mit Öl auf
Leinwand gemalt.
Das Bild ist in Eintönigem braungrau gehalten und wird etwas unterhalb der horizontallen Mitte durch eine schwarze
Horizontlinie geteilt. Auf der Erde sind viele einfache geometrische Körper, wahrscheinlich Steine
und Knochen, in den
Farben Grau, Weiß und Rot dargestellt. Durch diese Horizontlinie bekommt das Bild einen Tiefeneffekt. Hätte der Künstler
diese weggelassen, wäre nicht zwischen Himmel und Erde zu unterscheiden gewesen.
Oberhalb des Horizontes steigen schwarzbraune Rauchschwaden auf, die sich allerdings kaum vom braunen Himmel absetzen.

Der Künstler gehört seit 1925 zur Pariser Surrealistengruppe und will mit seinem Bild offenbar seine apokalyptische
Visionen abbilden, die er zehn Jahre vor den Atombombenabwürfenen auf Hiroshima auf Leinwand bringt.
Das Bild wirkt insgesamt sehr düster und leblos. Es strahlt aber durch die Schlichtheit viel ruhe aus und Vermittelt
zugleich eine apokalyptische Stimmung.

Das Bild scheint mir ein früher Hilferuf vom Künstler an die Vernunft der Menschheit zu sein, dass sich diese
nicht selbst in den Untergang treibt.
Da die Menschheit zu dieser Zeit anfing, eine sehr schwer kontrollierbare Energieform für sich zu nutzen, die Kernenergie.
Diese Unkontrolierbarkeit sollte nach dem Atombombenabwurf auf Hiroshima (1945) und der Katastrophe in Tschernobyl (1986),
aber spätestens durch die aktuellen Ereignisse in Fukushima Daiichi (2011) erkannt und bis auf weiteres nicht mehr als
Waffe oder in Reaktoren benutzt werden.

Auch wenn der Künstler nicht unbedingt die Kernenergie gemeint haben muss, so spricht er doch ein aktuelles Thema an.
