\section{Impressionismus}
\subsection{Merkmale, Inhalte, Anliegen}
\begin{itemize}
	\item Wirklichkeit malen wie wir sie wahrnehmen
	\item Bildthemen
	\begin{itemize}
		\item Landschaft, Licht das moderne Leben/Alltag, Menschen, Porträts, Stillleben,
			es sind positive heitere Bilder ohne Tragik
	\end{itemize}
	\item Bewegung, Veränderung der Dinge durch Lichtfall und Atmosphäre
	\item Eindrücke, Augenblicke
	\item Freiluftmalerei: \enquote{Plain air} ist Programm
\end{itemize}

\subsection{Malerische, Formale Merkmale}
\begin{itemize}
	\item Farbfläche setzen sich aus vielen nebeneinander gesetzten Pinselstrichen zusammen, Flirrende Farbflächen
	\item Auflösung der Gegenstandesformen
	\item Verwendung der reinen Farben dicht nebeneinander, mit Ziel Mischfarben erst im Auge des Betrachters
	entstehen zu lassen
	\item Augenfarbe\fxwarning{Rechtschreibung prüfen} der Lokalfarbe zu Gunsten der Erscheinungsfarbe
	dass heißt Dinge werden so gemalt wie sie im jeweiligen Licht erscheinen
	\item Farbe erhält einen Eigenwert dass heißt Farbe wird nicht mehr alleine dazu benutzt werden etwas abzubilden
	\item Der Impressionismus entsteht in Frankreich
\end{itemize}

\subsection{Paul Cézanne und Vincent van Gogh zwei wichtige Impulsgeber der Modernen}
\subsubsection{Paul Cézanne (1839--1906)}
Paul Cézanne wurde als Sohn eines Bankiers geboren und dabei sehr streng erzogen. Sehr bald erkannte er das er Kunst machen
wollte. Auf drängen des Vaters Studierte er in Paris Jura doch bald brach er ab und wollte sich in der Kunstakademie
einschreiben lassen, wurde aber abgelehnt. Nach der Absage kehrte er zurück zu seinem Vater und half diesem in der Bank.
Doch schon bald erkannte er wiederum, dass das ebenfalls nichts für ihn ist, und ging wieder nach Paris um sich in der
Kunstakademie einschreiben zulassen. Er wurde wieder abgelehnt. Da entschied er sich die Kunst selbst beizubringen
(Autodidakt). Um 1865 lernt er junge Impressionisten kennen und stellt mit diesen zusammen aus. Später lernte er Emilie
Hortense kennen, die er später heiratete. Aus dieser Beziehung ging der Sohn Paul heraus. Da er bei den Bürgern kaum
akzeptiert wurde, lebte er größtenteils in Armut, bis sein Vater starb und ihm alles vermachte. Mit 56 Jahren kamen seine
Bilder auch bei der breiten Bevölkerung an. 1906 starb er in Ex.

\subsubsection{Merkmale bei Paul Cézanne´s Malerei}
\begin{itemize}
	\item Erhöhter Abstraktionsgrad
	\item Perspektive wird fast aufgegeben
	\item Weniger Farben, diese aber fein variiert
	\item Bilder wirken wie aus einem Farbgewebe
	\item Farben wirken materiell schwer
	\item Formen erscheinen vereinfacht und geometrisiert
	\item Cézanne versuchte zu einer Art objektiven Wahrnehmung zu gelangen
	\item Cézanne untersuchte die Wirkung von Farbe und Farbnachbarschaften
\end{itemize}

\subsection{Vincent Willem van Gogh}
Vincent Willem van Gogh ist geboren 1853 in Holland. Es ist Sohn eines Pfarrers und hat mehrere Onkel die Kunsthändler
waren. Et begann eine Ausbildung als Kunsthändler, wurde aber entlassen, da er mit den Leuten fachsimpelte, anstatt zu
verkaufen. Nach der fehlgeschlagenen Ausbildung ging er als Laienprediger in das Kohlewerk Borinage bei Mons. Wurde aber
entlassen, weil er sein Geld verschenkte und wie die Arbeiter lebte. Er kehrte zurück zu seinen Eltern und begann dort mit
dem Male. Sein Bruder Theo (Kunsthändler) unterstützte ihn finanziell, damit er sich seine Ausrüstung leisten konnte. Bei
seinem Besuch in Paris lernte er impressionistische Maler kennen. Er war von diesem Stil sehr begeistert und änderte seinen
eigenen Stil. Als er sich mit einem Freund zusammentuen wollte, kam es zu anhaltenden Streits mit diesem. Dies nahm in so
mit das er sich selbst verletzte und sich ein Ohr abschnitt. Daraufhin merkte er das er eine schwere Psychose hatte und
ging in eine Heilanstalt. 1890 beging er Selbstmord.

\subsubsection{Merkmale bei Vincent van Gogh´s Malerei}
\begin{itemize}
	\item Malerei zeigt leuchtende Farben
	\item Geistige Malerei wirkt sehr bewegt
	\item Stark abstrahierte Formen
	\item In dem Bildern wir Freundfarbe\fxwarning{Rechtschreibung prüfen} eingesetzt
	\item Die Farbe und Kontraste erzeugen Emotionen
\end{itemize}

\ifDraft{\bigskip
\begin{center}
	\includegraphics[width=11cm]{files/images/malerei/1}
	\captionof{figure}{eigenes Bild}
\end{center}
}{}
