\section{Verzeichnis für Fachbegriffe}
\begin{description}
	\item[Bedeutungsperspektive:] Die Größenverhältnisse im Bild richten sich nach der Bedeutung
	einzelner Bildelemente (Mittelalter)
	\item[Zentralperspektive:] Gibt eine räumliche Situation wieder in größte möglicher Annäherung
	an das natürliche Vorbild (Merkmal: Fluchtpunkte)
	\item[Farb- und Luftperspektive:] Erzeugung von Tiefenräumlichkeit durch die Verschiebung von warmen Farbtönen
	in den Bildvordergrund und kalten Farbtönen in den Bildhintergrund.
	Hinzu kommt eine Atmosphärische Trübung, durch verblassen der Farben.
	\item[Lokalfarbe:] Ist eine Gegenstandfarbe, die dem Gegenstand eigene Farben.
	\item[Erscheinungsfarbe:] Ist die Farbe wie sie durch die atmosphärische Gegebenheit
	des Ortes und der Tageszeit erscheint (Impressionisten)
	\item[Ausdrucksfarbe:] Ist Farbe die eine seelische Empfindung zum Ausdruck bringt (Expressionismus)
	\item[Fremdfarbe:] Ist Farbe in einer malerischen Fläche,
	die am Objekt in der Wirklichkeit gar nicht vor kommt.
	\item[autonome Farbe:] Ist Farbe die weder die Funktion hat einen Gegenstand darzustellen
	noch einen Seelenzustand beschreibt.
	Es ist Farbe an sich. (abstrakte Kunst)
	\item[Ton in Ton Kontrast:] ruhiger Bildwirkung
	\item[Hell-Dunkel-Kontrast:] Theaterhaft, Theatralisch, Spannung
	\item[Kontrast der reinen Farben:] lebendig, ohne grell zu sein
	\item[Kontrast der getrübten Farben:] Vielfältigkeit
	\item[Primärfarbenkontrast (\textcolor{red}{rot}, \textcolor{yellow}{gelb}, \textcolor{blue}{blau}):]
	bunt und laut (knallt)
	\item[Komplementärfarben:] schrill
\end{description}
