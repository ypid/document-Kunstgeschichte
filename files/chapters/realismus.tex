\section{Realismus}		% (1830--1880)
\lettrine{D}{er} Realismus schlägt ein neues Kapitel der Kunstgeschichte auf.
Er bezieht sich nicht auf Vergangenes, sondern möchte die lebendige Wirklichkeit objektiv wiedergeben.

\renewcommand{\BildName}{Begräbnis von Ornans}
\subsection{Bildbeschreibung: \enquote{\BildName}}
\begin{center}
	\includegraphics[width=17cm]{files/images/Gustave}
	\captionof{figure}{\BildName ~(1849) von Gustave Courbet (1819--1877)}
\end{center}

% Seite 16 David (6804)
Das Bild zeigt eine Reihung von Personen, die Frauen rechts, die Männer links, die gelangweilt aber ordentlich herumstehen.
In der Mitte ist ein dunkles Loch, vor den Menschen in der Erde. Rechts neben dem Loch befindet sich ein weißer Hund, der
sich von den dunkel gekleideten Personen abhebt. Zur linken stehen zwei Bürdenträger der Kirche mit einem langen Stab, an
dessen Ende ein Kreuz hängt.

Insgesamt wirkt das Begräbnis sehr zeremoniell und unspektakulär denn die Farben und Flächen sind düster und ruhig
gehalten. Es steht die Wahrheit und nicht das Schöne im Vordergrund. Auch der Glaube, der hier durch die Vertreter der
Kirche symbolisiert wird, wird an den Rand gedrängt. Somit zeigt dieses Bild eine traurige und triste Alltagssituation auf
eine wahrheitsgemäße, realistische aber nicht perfektionierte Weise, weswegen es zur damaligen Zeit als Skandal angesehen
wurde.

Der Realismus entwickelt sich in Frankreich. Hauptvereter ist Gustave Courbet.

\begin{description}
	\item[Merkmale, Inhalte Anliegen:]
	  \begin{itemize}
	  	\item Darstellung der Wirklichkeit (reale Dinge, Menschen, Landschaften\dots )
	  	\item Nüchterne Darstellungsweise ohne Idealisierung
	  	\item Gezeigt werden soll das \enquote{Gewöhnliche}, \enquote{Alltägliche}
	  	\item Neues Thema \enquote{Arbeit}, oft mit sozialkritischer Tendenz
	  	\item Beginn der Freiluftmalerei \enquote{plain air}
	  \end{itemize}
	\item[Formale Merkmale:]
	  \begin{itemize}
	  	\item Die Malerei ist abbildhaft
	  	\item Farbpalette eher gedampft, getrübte Farbigkeit (keine grellen farben)
	  \end{itemize}
	\item[weitere Künstler:]
	  \begin{list}{}{}
	  	\item[Frankreich:]
		  \begin{itemize}
		  	\item Honoré Daumier
		  	\item Jean François Millet
		  	\item Jean-Baptiste Camille Corot
		  \end{itemize}
	  	\item[Deutschland:]
		  \begin{itemize}
		  	\item Adolph Menzel
		  	\item Wilhelm Leibl
		  	\item Fritz von Uhde
		  \end{itemize}
	  \end{list}
\end{description}









