\section{Barock}
\poemlines{4}
\settowidth{\versewidth}{Du siehst, wohin du siehst, nur Eitelkeit auf Erden.}
\begin{verse}[\versewidth]
	Du siehst, wohin du siehst, nur Eitelkeit auf Erden. \\
	Was dieser heute baut, reißt jener morgen ein; \\
	Wo jetzt Städte stehn, wird eine Wiese sein, \\
	Auf der ein Schäferskind wird spielen mit den Herden; \\
	Was jetzt und prächtig blüht, soll bald zertreten werden; \\
	Was jetzt so pocht und trotzt, ist morgen Asch und Bein; \\
	Nichts ist, das ewig sei, kein Erz, kein Marmorstein. \\
	Jetzt lacht das Glück uns an, bald donnern die Beschwerden. \\
\end{verse}
\attrib{Andreas Gryphius (1616--1664)}

\bigskip

\lettrine{D}{ie} Epoche des Barock ist vor allem durch den Konflikt der katholischen Kirche und der teilweise
reformierten Bevölkerung geprägt. Diese geht nun aktiv gegen die Reformation vor. Allgemein
kann diese Epoche als eine der schlimmsten Zeiten des Mittelalters gewertet werden, da Chaos,
Krieg, Hungersnöte, Krankheiten und der alltägliche Kampf ums überleben den Alltag des Volkes
bestimmte. Die größten Herausforderungen wahren der Bauernkrieg und der 30-jährige Krieg
(1618--1648). Während die Hexenverbrennung ihren Höhepunkt erreichte und die Schere zwischen
arm und reich immer weiter auseinander ging feierte der reiche Adel in Frankreich (es herrschte der
Absolutismus) ausgelassene Feste auf teuren Schlössern.

Kein Wunder also, dass Kunst ausschließlich von Adel und Kirche in Auftrag gegeben wurde und
nicht selten den Zweck der Selbstdarstellung hatte. Vor allem die Epoche des Rokoko (siehe unten)
zeigt die völlig abgehobene und realitätsferne Lebensweise des Adels. Die Künstler selbst kamen zu
großem Ansehen und es wurden erste Biografien von ihnen verfasst.

Das Ende des Barock wurde schließlich durch die französische Revolution eingeläutet. Getrieben
von den Aufklärungsgedanken ging das Volk nun gewaltsam gegen seine Situation und den dafür
verantwortlichen Adel vor. Ihre Forderung: Menschenrechte und das Motto: Freiheit, Gleichheit,
Brüderlichkeit. Die von Adel und Kirche in Auftrag gegebene Kunst, beispielsweise die Heilgenskulpturen in Kirchen wurden
missachtet und geschändet. Die Kunst sollten nun die neuen Werte
von Ehre und Moral darstellen.

\renewcommand{\BildName}{Grablegung Christ}
\subsection{Bildbeschreibung: \enquote{\BildName}}
\begin{center}
	\includegraphics[width=8cm]{files/images/bild-001}
	\captionof{figure}{\BildName ~(1602) von Michelangelo Merisi (1571--1610)}
\end{center}

Das Bild \enquote{\BildName} wurde 1602 im Auftrag von Francesco Vittrice von Michelangelo gemalt.

Es zeigt drei Frauen unterschiedlichen Alters die den fast nackten Jesus zur letzten Ruhe betten.
Dieser wird von zwei Männern niedergelegt, einer ist dabei im Vorder-,
der andere im fast unkenntlichen Hintergrund, der wie ein Schwarzes unbekanntes Loch wirkt.
%
Die drei Frauen scheinen mit dem Verlust ihres Heilands unterschiedlich umzugehen.
%
Das ganze ist in Bühnenhafter Beleuchtung dargestellt, die farbig aber eher schlicht gekleideten
Figuren sind also vor dem dunklen Hintergrund gut zu erkennen.
%
Als ganzes wirkt das Bild sehr bedrückend
und düster. Vor allem der Hintergrund zeigt
die Ungewissheit über das Leben nach dem
Tod auf.

\begin{list}{}{}
	\item[Technik:] Öl auf Leinwand (\SI{3 x 2}{\metre}), glatte Flächen,
realistisch, charakteristische Gesichtszüge,
fein gearbeitet, Proportionen stimmen, Jesus
(auch wegen dem weißen Tuch) am hellsten
dargestellt, bühnenhafte Beleuchtung
\end{list}

\renewcommand{\BildName}{Wildschweinjagd}
\renewcommand{\KuenstlerName}{Peter Paul Rubens}
\subsection{Bildbeschreibung: \enquote{\BildName}}
\begin{center}
	\includegraphics[width=14cm]{files/images/bild-000}
	\captionof{figure}{\BildName ~(1620) von \KuenstlerName ~(1571--1610)}
\end{center}

Das Bild \enquote{\BildName} um 1620 von \KuenstlerName ~gemalt
zeigt die Jagd zweier Gesellschaftsschichten auf ein Wildschwein.
Ein Teil der Jäger sitzt dabei auf Pferden und ist gut angezogen und mit Schwertern bewaffnet, die Anderen
-- vermutlich Bauern -- stellen dem Wild zu fuß und lediglich mit
Mistgabeln bewaffnet nach.
Auch viele Jagdhunde werden eingesetzt, manche wurden jedoch bereits durch das Wildschwein getötet.
Den Hintergrund bildet ein dichter Wald, seiner Farbe nach zu Urteilen ist es Herbst.
Durch eine Lichtung scheint die Herbstsonne und taucht die Szene in goldenes Licht.
Außerdem ist ein Stück blauer Himmel zu sehen.
Die Flächen sind gut durchgearbeitet und belebt was das Bild naturgetreu, detailliert und realistisch wiedergibt.
Die Bewegung der Jäger und ihrer Pferde machen die Handlung dramatisch und angespannt,
es handelt sich also um eine Momentaufnahme.
Das Bild löst bei mir nicht gerade einen Jagdrausch aus, vielmehr bemitleidet man das Wildschwein,
das aufgrund seiner großen Caninus (Eckzähne) als Verkörperung des durch und durch
Bösen den Sündenbock spielen und somit die Welt in seiner (für es) schönsten Jahreszeit verlassen muss.
