\begin{itemize}
	\item politische Situation
	\begin{itemize}
		\item Bürgertum, Mediei
		\item Stadtrepublik, Mitbestimmungsrecht
		\item Reichtum durch Entdeckung neuer Länder/Welten
		\item Stadthandel
		\begin{itemize}
			\item Wirtschaftlicher Aufschwung
		\end{itemize}
		\item Fall Konstantinopels bringt Flüchtlinge und Kulturelles, Gelehrte
		\item Kirche verliert an Einfluss
	\end{itemize}
	\item Entdeckungen/Erfindungen
	\begin{itemize}
		\item Buchdruck
		\item Bibelübersetzung
		\item Verbesserung der Verkehrswege
		\item Seekarten, Kompass
		\begin{itemize}
			\item Entdeckung der Neuen Welten
		\end{itemize}
		\item Kopernikus (Heliozentrisches Weltbild)
	\end{itemize}
	\item Geisteshaltung/Philosophisches Denken
	\begin{itemize}
		\item Reformation (Befreiung von Bevormundung)
		\item Wandel vom Christrosentischen Weltbild in ein anthropozentrisches Weltbild
		\begin{itemize}
			\item Menschen \entspricht Mittelpunkt
		\end{itemize}
		\item Humanismus (Entdeckung des Menschen)
		\item Allgemein Bildung der Bevölkerung \entspricht Ziel
		\item Anknüpfung an die Antike
		\item Erforschen der Natur/Mensch
	\end{itemize}
	\item Funktion der Kunst/Stellung des Künstlers Auftraggebers
	\begin{itemize}
		\item Wandel vom Handwerker zum freischaffendem Künstlers
		\item Künstler ist Forscher
		\item Auftraggeber: Päpste und reiche Bürger
	\end{itemize}
	\item Entdeckungen/Neuerungen
	\begin{itemize}
		\item Zentralperspektive
		\item Farb- und Luftperspektive
		\item Ölmalerei
	\end{itemize}
	\item Merkmale Kunst/Bildthemen
	\begin{itemize}
		\item Selbstbildniss, Akt
		\item Landschaft als Umrandung
		\item Naturnahe Darstellung
		\item Neben christlichen Bildthemen mythologische Bildthemen
		\item Maße und Proportionen
		\item Darstellung von Raum
	\end{itemize}
\end{itemize}