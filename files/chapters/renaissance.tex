\section{Renaissance}
\begin{tabular}{ll}
	Italien	& 1400--1500\\
	Norden	& 1500--1580\\
\end{tabular}

\renewcommand{\BildName}{Die Vermählung Mariä}
\renewcommand{\KuenstlerName}{Jacques-Louis David}
\subsection{Bildbeschreibung: \enquote{\BildName}}
\begin{center}
	\includegraphics[width=8cm]{files/images/bild-003}
	\captionof{figure}{\BildName ~(1504) von Raffael (1483--1520)}
\end{center}

\begin{itemize}
	\item Was
	\begin{itemize}
		\item Eine Verlobung, vier Frauen wenden sich der Maria zu, vier dem Mann.
		\item Der Priester steht im Mittelpunkt
		\begin{itemize}
			\item Göttliches, er verbindet beide Seiten.
		\end{itemize}
		\item Elegantes Gewänder, sehr Farbenfroh gekleidete Personen. Dadurch entsteht Tiefe
		\item Fluchtpunkt ist der Durchblick durch die Kirche
		\item Eine Kirche im Hintergrund auf die alles zuläuft
		\item Symmetrischer Bildaufbau in Zentralperspektive
	\end{itemize}
	\item Wirkung
	\begin{itemize}
		\item Es wirkt ruhig und harmonisch durch die gedeckten Farben
		\item Der Kuppelbau bringt Stabilität und Sicherheit
		\begin{itemize}
			\item die Kirche sorgt für Ordnung
		\end{itemize}
		\item Der Rhythmus der Tempelstufen stellt eine Verbindung zwischen den Menschen und der Kirche her
	\end{itemize}
	\item Merkmale
	\begin{itemize}
		\item Verlobung findet auf der Erde statt, das Biblische spiegelt sich durch die Landschaft im Hintergrund wieder
	\end{itemize}
\end{itemize}

\begin{itemize}
	\item politische Situation
	\begin{itemize}
		\item Bürgertum, Mediei
		\item Stadtrepublik, Mitbestimmungsrecht
		\item Reichtum durch Entdeckung neuer Länder/Welten
		\item Stadthandel
		\begin{itemize}
			\item Wirtschaftlicher Aufschwung
		\end{itemize}
		\item Fall Konstantinopels bringt Flüchtlinge und Kulturelles, Gelehrte
		\item Kirche verliert an Einfluss
	\end{itemize}
	\item Entdeckungen/Erfindungen
	\begin{itemize}
		\item Buchdruck
		\item Bibelübersetzung
		\item Verbesserung der Verkehrswege
		\item Seekarten, Kompass
		\begin{itemize}
			\item Entdeckung der Neuen Welten
		\end{itemize}
		\item Kopernikus (Heliozentrisches Weltbild)
	\end{itemize}
	\item Geisteshaltung/Philosophisches Denken
	\begin{itemize}
		\item Reformation (Befreiung von Bevormundung)
		\item Wandel vom Christrosentischen Weltbild in ein anthropozentrisches Weltbild
		\begin{itemize}
			\item Menschen \entspricht Mittelpunkt
		\end{itemize}
		\item Humanismus (Entdeckung des Menschen)
		\item Allgemein Bildung der Bevölkerung \entspricht Ziel
		\item Anknüpfung an die Antike
		\item Erforschen der Natur/Mensch
	\end{itemize}
	\item Funktion der Kunst/Stellung des Künstlers Auftraggebers
	\begin{itemize}
		\item Wandel vom Handwerker zum freischaffendem Künstlers
		\item Künstler ist Forscher
		\item Auftraggeber: Päpste und reiche Bürger
	\end{itemize}
	\item Entdeckungen/Neuerungen
	\begin{itemize}
		\item Zentralperspektive
		\item Farb- und Luftperspektive
		\item Ölmalerei
	\end{itemize}
	\item Merkmale Kunst/Bildthemen
	\begin{itemize}
		\item Selbstbildniss, Akt
		\item Landschaft als Umrandung
		\item Naturnahe Darstellung
		\item Neben christlichen Bildthemen mythologische Bildthemen
		\item Maße und Proportionen
		\item Darstellung von Raum
	\end{itemize}
\end{itemize}