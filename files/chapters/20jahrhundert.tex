\section{Die Kunst des 20. Jahrhundert}
\subsection{Die Fauvisten (französische Expressionisten)}
\begin{itemize}
	\item 1905 im Pariser Herbstsalon
	\begin{itemize}
		\item Fauvisten stellen ihre Bilder aus
	\end{itemize}
	\item Es kommt zum Skandal, durch einfache Malerei und \enquote{schreiende} Farben
	\item gemeinsame Grundhaltung
	\begin{itemize}
		\item starke Farben
		\item Vereinfachung des Gegenständlichen
		\item Verzicht auf Körpermodellierung
	\end{itemize}
\end{itemize}

\renewcommand{\BildName}{Corbeille d'oranges}
\renewcommand{\KuenstlerName}{Henri Matisse}
\begin{center}
	\includegraphics[width=10cm]{files/images/Bilder-4_2.jpg}
	\captionof{figure}{\BildName ~(1912/13) von \KuenstlerName ~(1869--1954)}
\end{center}

\ifDraft{}{\newpage}	%% Postoptimierung

\subsection{Brücke (norddeutsche Expressionisten)}
\begin{itemize}
	\item Heine Gruppe von Architektenstudenten der Dresdner Hochschule
	\begin{itemize}
		\item Künstlergemeinschaft \enquote{Brücke} (1905--1913)
	\end{itemize}
	\item malen, seelisch, psychische Motive
	\item das Hässliche, ursprüngliche Schönheit der Natur
	\item wenig Wert auf optische Erscheinung von dem Sinn der Dinge zu erfassen und wiederzugeben.
	\item bewusste Vergröberung $\rightarrow$ holzschnittartige Gestik
\end{itemize}

\subsection{Blaue Reiter (süddeutsche Expressionisten)}
\begin{itemize}
	\item Kandinskys Austritt aus der \enquote{Künstlervereinigung Mürchen} aufgrund seiner abstrakten Malweise
	\begin{itemize}
		\item gründet Künstlergruppe: Der blaue Reiter \\
		eigene Ausstellung $\rightarrow$ löst sich am beginn des 1. Weltkrieges auf
	\end{itemize}
	\item Versuch die innere und geistige Welt in ihren Bilder auszudrücken
	\begin{itemize}
		\item Befreiung der Kunst von Stofflichen Erlebnissen, Abstraktion
	\end{itemize}
	\item meditative Expressionisten
\end{itemize}

\renewcommand{\BildName}{Murnau mit Kirche}
\renewcommand{\KuenstlerName}{Wassily Kandinsky}	% Frau Preisler scheib sicher nicht, es ist von Sogh oder so
\begin{center}
	\includegraphics[width=10cm]{files/images/Bilder-5_1.jpg}
	\captionof{figure}{\BildName}
\end{center}

\subsection{Kubismus}
\begin{itemize}
	\item Frühkubismus (1906--1910) wichtige Vertreter: Pablo Picasso und Georges Braque
	\item Merkmale
	\begin{itemize}
		\item Studien an afrikanischen Masken
		\item hochgezogener Horizont, blockhafte Formen (geometrisierung)
		\item Aufgaben des fixierten Blickpunktes, von dem aus das Bild angeschaut wird
	\end{itemize}
	\item \enquote{Papier collé} (ins Bild geklebte Realitätsfragmente)
	\item Bruch der ästhetischen Vorstellungen und Regeln was die Behandlung des Raumes
	und den Ausdruck menschlichen Empfindens angeht
	\item Figurenbilder, Landschaften, Stillleben
\end{itemize}

\renewcommand{\BildName}{Häuser in L’Estaque}
\renewcommand{\KuenstlerName}{Georges Braque}
\begin{center}
	\includegraphics[width=10cm]{files/images/braque.jpg}
	\captionof{figure}{\BildName ~(1908) von \KuenstlerName ~(1882--1963)}
\end{center}

\subsection{Analytischer Kubismus}
\begin{itemize}
	\item gekennzeichnet durch Analyse des Gegenstandes
	\item man betrachtet Objekte aus verschiedenen Perspektiven und fügt das
	Gesehene in Neukompositionen zusammen
	\item Neu: Darstellung mehrerer Aspekte in einem Bild
	\item Beziehungen der Dinge und die Relation zum Umraum wird wichtig
	\begin{itemize}
		\item Folgen: Reduktion auf grau, blau, Ockertöne und auf einfache Motive (Krüge, Gläser)
	\end{itemize}
	\item neue Möglichkeit für den Betrachter zu interpretieren
\end{itemize}

\renewcommand{\BildName}{Violine und Krug}
\begin{center}
	\includegraphics[width=9cm]{files/images/braque-violine_krug-1910}
	\captionof{figure}{\BildName ~(1910) von \KuenstlerName ~(1882--1963)}
\end{center}

%\subsection{Synthetischer Kubismus}
%\begin{itemize}
%	\item aus frei erfundenen, abstrakten Bildelementen wird der Bildgegenstand zusammengefügt
%\end{itemize}

\subsection{Orphismus}
\begin{itemize}
	\item auch \enquote{Orphischer Kubismus} genannt (steht ihm nahe)
	\item stammt von dem Schriftsteller Guillaume Apollinaire ab
	\item Anspielung auf den mythischen Sänger Orpheus
	\begin{itemize}
		\item steht für neue politische Wirklichkeit
	\end{itemize}
	\item kristallene Formen
	\item Farben spielen zentrale Rolle
	\item Spiel mit Lichtbrechungen und Winkeln
	\item gegen später fast nur reine Farben, Fenstermalerei
	\begin{itemize}
		\item Ziel der Bewegung: innerliche, gemeinverständlich, politische Sicht zu erlangen
	\end{itemize}
\end{itemize}

\renewcommand{\BildName}{Les Fenetres simultanée sur la ville}
\renewcommand{\KuenstlerName}{Robert Delaunay}
\begin{center}
	\includegraphics[width=10cm]{files/images/Robert-Delaunay.jpg}
	\captionof{figure}{\BildName ~(1912) von \KuenstlerName ~(1885--1941)}
\end{center}

\subsection{Futurismus}
\begin{itemize}
	\item eine Revolte junger italienischer Literaten und Künstler gegen alles Traditionelle und Konservative
	\item Schönheit der Geschwindigkeit -- \enquote{Meisterwerke müssen aggressiv wirken}
	\item Zerstreuung und Verschmelzung der Details
	\item zeitliche Abläufe werden parallel dargestellt
	\item Verschiebung und Zerlegung der Gegenstücke
	\item Giacomo Balla: \enquote{Ein Kind läuft über den Balkon} (1912)
\end{itemize}
