\begin{itemize}
	\item politische Situation
	\begin{itemize}
		\item Feudalismus (Kirche, Adel, Bauerntum)
		\item Wandel in der Spätgotik
		\begin{itemize}
			\item zunehmende Bedeutung der Städte
			\item Machtkampf Kirche/Adel
			\item Pestepedemien
		\end{itemize}
	\end{itemize}
	\item Entdeckungen/Erfindungen\\
	1400/50
	\begin{itemize}
		\item Papierherstellung
		\item Buchdruckkunst
	\end{itemize}
	\item Merkmale gotischer Kunst/Bildthemen
	\begin{itemize}
		\item Christliche Bildinhalte
		\item Symbolsprache
		\item Bedeutungsperspektive
		\item flächige Malweise: Goldgrund, Raumlosigkeit
	\end{itemize}
	\item Technik
	\begin{itemize}
		\item Fresko, Buchmalerei
		\item Spätgotik
		\begin{itemize}
			\item  mehr Realitätsnahe, geahnte Perspektive
		\end{itemize}
	\end{itemize}
	\item Geisteshaltung/Philosophisches Denken
	\begin{itemize}
		\item Scholastik (mittelalterliche Beweisführung)
		\item Wissen ist in den Klöstern beheimatet
		\item Ideen des Humanismus entwickeln sich
		\item Interesse am Irdischen entsteht
	\end{itemize}
	\item Funktion der Kunst/Stellung des Künstlers Auftraggebers
	\begin{itemize}
		\item Sakrale Kunst
		\item Kirche als Auftragsgeber
		\item Künstler = Handwerker
		\item Spätgotik
		\begin{itemize}
			\item Kirche und reiche Bürger als Auftraggeber
			\item Künstler verlieren ihre Anonymität
		\end{itemize}
	\end{itemize}
\end{itemize}
