\section{Avantgarde}
\lettrine{N}{eben} den Künstlern, die sich dem Geschmack der Zeit und dem Konventionellen fügen, gibt es zunehmend
Künstler die eigene
Wege gehen. Das neu gewonnene Selbstvertrauen in die eigene Persönlichkeit, ausgelöst durch den wachsenden Wunsch nach
Freiheit und Individualität nach der französischen Revolution, führt dazu, dass die Künstler sich zunehmend alleine ihren
persönlichen künstlerischen Gewissen verantwortlich fühlen.

Sie malen was sie für wichtig halten. Hierbei entsteht zunehmend eine Kluft zwischen Künstler und Publikum. Der Künstler
wird häufig verkannt, nicht verstanden. Er rückt an den Rand der Gesellschaft.

\ifDraft{
\subsection{Vergleichende Bildbeschreibung}

\renewcommand{\BildName}{Goldgrube}
%\subsection{Bildbeschreibung: \enquote{\BildName}}
\begin{center}
	\includegraphics[width=17cm]{files/images/Bilder-3_2}
	\captionof{figure}{\BildName ~(2007) von Neo Rauch (1960)}
\end{center}

\renewcommand{\BildName}{Selbstbildnis am Abgrund}
%\subsection{Bildbeschreibung: \enquote{\BildName}}
\begin{center}
	\includegraphics[width=12cm]{files/images/Bilder-2_4}
	\captionof{figure}{\BildName ~(1848) von Gustave Courbet (1819--1877)}
\end{center}
}{}
