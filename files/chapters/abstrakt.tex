\section{Die abstrakte- ungegenständliche Bildwelt}
\begin{itemize}
	\item Zentral ist eine große auf die Spitze gestellte Rautenform in türkis-blau
	\item links unten ist ein Kreis (lila--blau) mit zeigerförmigen Gebilden, die über die Kreisform
	hinausragen und eine Verbindung zur Raute darstellen.
	\item Oben rechts ist eine Ballung von verschieden großen Kreisen in unterschiedlichen
	\item Man sieht viele geometrische Formen in bunten Farben auf dem Bild verteilt.
	Sie sind auf einen rötlich- braunen Grund gesetzt, der keine einheitliche Fläche ist,
	sondern in der Mitte heller ist als außen.
	\item Das Bild wirkt technisch, weil die geometrischen Formen gezielt gesetzt wurden und eine
	klare Abgrenzung der Formen zum Umraum aufweisen.
	\item Es wirkt statisch, weil die Raute in der Mitte zentriert steht
	\item Das auf dem Bild dargestellte wirkt wie eine Einheit, da keine der Formen zum Rand hin abgeschnitten ist.
	\item Interpretation
	\begin{itemize}
		\item der Künstler hat eventuell die Schönheit in abstrakten Formen gesucht (kein Abbild)
	\end{itemize}
	\item Synästhesie: verschiedene Sinne ineinander übertragen; \zB Musik und Malerei
\end{itemize}
