\section{Klassizismus und Romantik}
\lettrine{K}{lassizismus} und Romantik sind zwei Kunstrichtungen, die sich fast zeitgleich im Anschluss an das barocke
Zeitalter
entwickeln. Der Klassizismus etwas früher, ab 1750 nach Christus die Romantiker ab 1800 nach Christus.

\renewcommand{\BildName}{Der Tod des Marat}
\renewcommand{\KuenstlerName}{Jacques-Louis David}
\subsection{Bildbeschreibung: \enquote{\BildName}}
\begin{center}
	\includegraphics[width=10cm]{files/images/bild-004}
	\captionof{figure}{\BildName ~(1793) von \KuenstlerName ~(1748--1825)}
\end{center}

Das Bild \enquote{\BildName} von \KuenstlerName ~entstand während der französischen Revolution und hat einen
wichtigen historischen Hintergrund.


Jean-Paul Marat war Mittglied des Nationalkonvents, Journalist der
radikalen Zeitung \enquote{L'Ami du Peuble}, Mitglied des
radikalen Jakobinerklubs und Befürworter der Hinrichtung des Königs
-- im Großen und Ganzen also ein sehr radikaler Revolutionär.


Doch die Revolution hatte auch ernst zu nehmende Feinde und so wird er
1793 von Anarchistin Charlotte Corday während des Verfassens seiner
Texte in der Badewanne erstochen.


Sie hinterlässt einen Brief, auf dem sie ihre Tat rechtfertigt: Sie habe
einen Mann getötet, um hunderttausend zu retten. Damit spielt sie auch
auf das sogenannte Septembermasaker an, bei dem 1200 Gefangene
abgeschlachtet wurden, darunter waren aber eher wenige
Revolutionsfeinde. Da Marat ein Großteil dazu beigetragen hatte, die
Menschen auf die Gefangenen zu hetzten hatte er den Tod an vielen
Zivilisten, die nur kleine Verbrechen begangen hatten zu verantworten.


Nach seinem Tod wird er von seinem Freund Jacques-Louis David
aufgefunden, der ihn am Vortag noch besucht hatte. Er erhielt den
Auftrag das Bild des toten Märtyrers für Propagandazwecke festzuhalten
und so stellte er es innerhalb vier Monate fertig.


Dabei idealisierte er ein paar Sachverhalte, so ist die Hautkrankheit,
die Marat zu längeren Bädern zwang nicht dargestellt.
Stattdessen wird seine Körperhaltung an die anderer Märtyrer angeglichen.
Trotz dieser relativ kurzen Bearbeitungszeit zählt dieses Werk zu den Wichtigsten Davids.

Auffällig ist die Wahl des Hochformats, da das eigentliche Motiv nur die
untere Hälfte der Fläche in Anspruch nimmt. Die restliche obere Hälfte
ist größtenteils mit schwarzem Hintergrund gefüllt.
Dies bewirkt vor allem die bühnenhafte Beleuchtung,
welche den Toten in ein Milchig weißes Licht taucht.

Der Holzkasten mit der Aufschrift \enquote{À MARAT} erinnert an einen Grabstein,
also ein Andenken.

\renewcommand{\BildName}{Schiffbruch im Eismeer}
\subsection{Bildbeschreibung: \enquote{\BildName}}
\begin{center}
	\includegraphics[width=12cm]{files/images/bild-005}
	\captionof{figure}{\BildName ~(1823) von Caspar David Friedrich (1774--1840)}
\end{center}

\begin{itemize}
	\item das Bild zeigt ein Trümmerfeld aus bräunlichen Eisplatten ein im Eismeer gesunkenes Schiff,
	in der Mitte ein Goldfelsen
	\item links einen winzigen Mast, überall Holzsplitter
	\item die Sonne scheint von oben rechts
	\item exakt mit Öl auf eine Leinwand gemalt, kaum Rundungen
	\item vermehrt blasse Blautöne, feine Farbflächen
	\item Riesen Eisplatten wirken majestätisch, das Schiff dagegen winzig und klein
	\begin{itemize}
		\item spiegelt den Kontrastverlust der Menschheit über die Industrialisierung wieder
		\item die Natur ist nicht kontrollierbar
	\end{itemize}
	\item Casper David Friedrich: großer dt. Romantiker
\end{itemize}

\renewcommand{\BildName}{Die Erschießung der Aufständischen}
\subsection{Bildbeschreibung: \enquote{\BildName}}		%% Seite -12- David (6801.jpg)
\begin{center}
	\includegraphics[width=14cm]{files/images/bild-006}
	\captionof{figure}{\BildName ~(1814) von Francisco de Goya (1746--1828)}
\end{center}

\begin{itemize}
	\item rechts Soldaten die Waffen auf knienden Aufständischen richtet, die links im Bild zu sehen sind
	\item Mittelpunkt ist ein zum Himmel gerichteter Mann in weißer Kleidung.
	Er hat die Hände gehoben, hinter ihm verstecken sich die Ängstlichen.
	\item sonst nur bräunliche, gedeckte Farben
	\item Bild zeigt die Eroberung einer Stadt durch Napoleon.
	\item Man fühlt mit den Aufständischen mit da man alles aus der Perspektive der Soldaten betrachtet,
	dass heißt man sieht von den bewaffneten Soldaten die Rücken,
	aber von den Gefangenen die flehenden und ängstlichen Gesichter.
	\item nicht eindeutig der Klassik oder Romantik zuzuordnen.
\end{itemize}

\subsection{Klassizismus (1750--1840)}
\subsubsection{Merkmale, Inhalte, Anliegen}
\begin{itemize}
	\item geistiger Hintergrund ist die Philosophie der Aufklärung
	\item Kunst soll im Dienst anerkannter Werte stehen
	(\zB Ehre, Moral, Gerechtigkeit, Pflichtgefühl, Patriotismus,\dots)
	\item erneute Rückbesinnung auf die Antike und Renaissance (die gute Form, Perfektion, Schönheit)
\end{itemize}

\subsubsection{Malerische, Formale Merkmale}
\begin{itemize}
	\item exakte Linienführung, schöne, perfekte Formen
	\item klare Komposition und Form
\end{itemize}

\subsection{Romantik (1800--1860)}
\subsubsection{Merkmale, Inhalte, Anliegen}
\begin{itemize}
	\item Gegenposition zur verstandsbetonten Kunsthaltung des Klassizismus
	\item Emotionale innere Werte sind maßgeblich (seelisches kommt zum Ausdruck)
	\item Hang zu Idylle und Weltflucht,
	Traum von einer besseren Weltflucht
	\item Oft religiöse, übergeordnete Botschaften
	(Grundlage bildet eine religiöse Erneuerung nach den Ereignisen der französischen Revolution)
	\item freie Bilderfindungen, Fantasie
\end{itemize}

\subsubsection{Malerische, Formale Merkmale}
\begin{itemize}
	\item Stimmungen werden durch ausgewählte Farbgebunden erzeugt
	\item die Natur, Landschaft, Tierdarstellungen werden in überhöhter Weise dargestellt
	\item Bilder transportieren einen übergeordneten Inhalt (symbolisch)
\end{itemize}
