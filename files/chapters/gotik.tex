\section{Die Gotik}
\begin{tabular}{ll}
			& \textasciitilde 1200\\
	Italien	& 1400\\
	Norden	& 1500\\
\end{tabular}

\renewcommand{\BildName}{La Pietà d'Avignon}
\renewcommand{\KuenstlerName}{Enguerrand Quarton}
\subsection{Bildbeschreibung: \enquote{\BildName}}
\begin{center}
	\includegraphics[width=14cm]{files/images/bild-002}
	\captionof{figure}{\BildName ~(1454) von \KuenstlerName ~(1410--1466)}
\end{center}

Das Bild \enquote{\BildName} aus der Kartause von Villeneuve-lès-Avignon, wurde 1454 von dem französischem
Künstler \KuenstlerName, als Auftragsarbeit gemalt.
Das Thema des Bildes ist der Tod und damit verbundene Trauer um Jesus Christus.

Der nur mit einem Lendenschutz bedeckte Körper befindet sich auf dem Schoß der heiligen Maria, in der Mitte des Bildes.
Die in einen warmes Blau gekleidete Maria hat die Hände gefaltet und blickt voller Trauer und Liebe auf ihren Sohn hinab.
Zur Rechten und Linken Jesus befinden sich zwei trauernde Menschen, die mit geneigtem Kopf und gekrümmten Körpern um ihren
Heiland weinen. Im Hintergrund dieser familiären Trauerszene befindet sich ein kleiner Tempel der Geborgenheit ausstrahlt.

Abseits der Bildmitte am linken Rand kniet ein betender Mönch, der sich durch sein weißes Gewand und seine starre,
gefühllose und aufrechte Haltung, von dem Trauerspiel von Christus, deutlich abgrenzt. Hinter ihm wirkt die Stadt
Jerusalem wie eine kalte Festung, die ihm Macht verleiht. Der Mönch symbolisiert gleichzeitig auch den Mann, der das Bild
in Auftrag gab.

Der Hintergrund des Bildes erstrahlt in einem Goldton. Das Bild ist mit einer Mischtechnik auf Holz gemalt.
Es wurden hauptsächlich warme, gedeckte Farben verwendet wie beispielsweise das Blau von Marias Kleidung.
Die Warme der Farben verstärkt den Eindruck von einem liebevollen und familiären Miteinander der Personen,
die sehr nahe und gefühlsecht wirken, da sie realistisch, fast fotografiert dargestellt sind.

\begin{itemize}
	\item politische Situation
	\begin{itemize}
		\item Feudalismus (Kirche, Adel, Bauerntum)
		\item Wandel in der Spätgotik
		\begin{itemize}
			\item zunehmende Bedeutung der Städte
			\item Machtkampf Kirche/Adel
			\item Pestepedemien
		\end{itemize}
	\end{itemize}
	\item Entdeckungen/Erfindungen\\
	1400/50
	\begin{itemize}
		\item Papierherstellung
		\item Buchdruckkunst
	\end{itemize}
	\item Merkmale gotischer Kunst/Bildthemen
	\begin{itemize}
		\item Christliche Bildinhalte
		\item Symbolsprache
		\item Bedeutungsperspektive
		\item flächige Malweise: Goldgrund, Raumlosigkeit
	\end{itemize}
	\item Technik
	\begin{itemize}
		\item Fresko, Buchmalerei
		\item Spätgotik
		\begin{itemize}
			\item  mehr Realitätsnahe, geahnte Perspektive
		\end{itemize}
	\end{itemize}
	\item Geisteshaltung/Philosophisches Denken
	\begin{itemize}
		\item Scholastik (mittelalterliche Beweisführung)
		\item Wissen ist in den Klöstern beheimatet
		\item Ideen des Humanismus entwickeln sich
		\item Interesse am Irdischen entsteht
	\end{itemize}
	\item Funktion der Kunst/Stellung des Künstlers Auftraggebers
	\begin{itemize}
		\item Sakrale Kunst
		\item Kirche als Auftragsgeber
		\item Künstler = Handwerker
		\item Spätgotik
		\begin{itemize}
			\item Kirche und reiche Bürger als Auftraggeber
			\item Künstler verlieren ihre Anonymität
		\end{itemize}
	\end{itemize}
\end{itemize}
