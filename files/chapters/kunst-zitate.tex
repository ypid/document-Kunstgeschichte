\section{Kunstzitate}
\enquote{Kunst gibt nicht das Sichtbare wieder, sie macht sichtbar.}
\signed{Paul Klee (1879--1940)}

\bigskip
\bigskip
%%Next

\enquote{Kunst ist die Befreiung aus den Fesseln des Faktischen.}
\signed{Andreas Tenzer (1954)}

\bigskip
\bigskip
%%Next

\enquote{Die Kunst ist zwar nicht das Brot, wohl aber der Wein des Lebens.}
\signed{Jean Paul (1763--1825)}

\bigskip
\bigskip
%%Next

\enquote{Die Kunst darf nie populär sein wollen. Das Publikum muß künstlerisch werden.}
\signed{Oscar Wilde (1854--1900)}

\bigskip
\bigskip
%%Next

\enquote{Das Schönste, was wir erleben können, ist das Geheimnisvolle. Es ist das Grundgefühl, das an der Wiege von wahrer
Kunst und Wissenschaft steht. Wer es nicht kennt und sich nicht mehr wundern, nicht mehr staunen kann, der ist sozusagen
tot und sein Auge ist erloschen.}
\signed{Albert Einstein (1879--1955)}

\bigskip
\bigskip
%%Next

\enquote{Was wäre das leben, hätten wir nicht den Mut, etwas zu riskiren}
\signed{Vincent Willem van Gogh (1890--1978)}

\bigskip
\bigskip
%%Next

\enquote{Jeder möchte die Kunst verstehen. Warum versucht man nicht, die Lieder eines Vogels zu verstehen? Warum liebt man
die Nacht, die Blumen, alles um uns herum, ohne es durchaus verstehen zu wollen? Aber wenn es um ein Bild geht, denken die
Leute, sie müssen es \enquote{verstehen}}
\signed{Pablo Picasso (1881--1973)}

\bigskip
\bigskip
%%Next

\enquote{Wir wissen alle, daß Kunst nicht Wahrheit ist. Kunst ist eine Lüge, die uns die Wahrheit begreifen lehrt,
wenigstens die Wahrheit, die wir als Menschen begreifen können. Der Künstler muß wissen, auf welche Art er die anderen von
der Wahrhaftigkeit seiner Lügen überzeugen kann.}
\signed{Pablo Picasso (1881--1973)}

\bigskip
\bigskip
%%Next

\enquote{Selbstverständlich ist die Kunst ihrem Wesen nach verwerflich! Und überflüssig! Und asozial, subversiv,
gefährlich! Und wenn sie das nicht ist, dann ist sie weiter nichts als Falschgeld, leere Hülle, Kartoffelsack \dots}
\signed{Jean Dubuffet (1901--1985)}
